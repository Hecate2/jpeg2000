\section{生成数据流}
将生成的stream和Context储存为二进制文件,每一个整数用4个字节保存。context的范围为0-18,stream的范围为0-255。由于分块后tile的大小为1024*1024,故小波变换后各频带“图像”的大小小于1024。因此,可选用大于1024的整数作为分隔符。不同tile之间用2051分隔,不同频带之间用2050分隔,不同block(64*64)之间用2049分隔,CX和stream之间用2048分隔。\textit{代码见函数image\_entropy, tile\_entropy, block\_entropy。}\\ 
\indent 由于小波变换后的各频带"图像"大小并不是64的整数,在进行内嵌区段编码前需要补零,才能将图像分割为m*n个64*64的block。因此,在数据流中还需要保存“图像”大小。这样,解码时将补零后的图像恢复成原先的大小。